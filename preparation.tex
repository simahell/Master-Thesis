\chapter{準備}
\section{形式手法}
形式手法とは、これはシステムの安全性を数学的に証明する方法であり、システム設計を手助けする方法の一つである。
一般に産業界で用いられる方法は形式手法とは異なり、作成したシステムに様々な入力を与えて検査する方法である。
この方法では低コストで検査することが可能だが、テストのための入力を人が考えるためシステムに不具合が残る可能性がある。
一方で、形式手法は数学的な証明であるため、そういった不具合を無くすことができる。
また、この手法は専用の検査ツールを利用することで自動化が可能であり、検査に掛かる手間を大幅に減らすことが可能である。

実際にシステム開発に形式手法を利用する際には、以下の四つのステップの手順を行う。
\begin{enumerate}
\item 設計したシステムをセキュリティモデル(\ref{sec:SecurityModel}参照)として表現する。
\item 用意したセキュリティモデルを入力として形式手法ツールを実行し、設計の漏れや仕様上の不具合を検出する。
\item 2.で検出された漏れや不具合を参考にシステムの設計を見直し、その修正をセキュリティモデルに適用する。
\item 2.以降を繰り返す。
\end{enumerate}

\subsection{セキュリティモデル}
\label{sec:SecurityModel}
セキュリティモデルは形式手法における入力にあたり、以下の三要素で構成される。
\begin{itemize}
\item 対象のシステムの構造
\item 脅威モデル
\item 安全性要件
\end{itemize}

\subsection{Alloy Analyzer}
Alloy Analyzerは形式手法による解析ツールの一つである。
検査対象のシステムのセキュリティモデルをAlloyという専用の言語で記述し、これを入力として実行する。
その結果としては二通りの出力を得ることができ、まずは、そのシステムが設計上取りうる状態の例が表示され、これを利用することでシステムが設計者の意図していない挙動を行っていないかを確認でき、設計上の漏れを防ぐことができる。
また、安全性要件を満たさない危険な状態を検索することも可能であり、これによりシステムが設計通りに動作したとしても危険な状態に陥ってしまうような不具合を見つけることができる。

また、Alloy Analyzerが他の形式手法ツールとは異なる点として、実行結果を図として得られるため、より直観的な利用が可能であることが挙げられる。
さらに、Alloy Analyzerは汎用的なJavaを用いて実装されているため、環境の構築はJavaのインストールのみで済み簡単に利用することができる。

\section{時相論理}
時相論理とは、

\subsection{線形時間論理}


\subsection{分岐論理}


\section{キャッシュ}


\subsection{Browser Cache Poisoning攻撃}

\section{中継者}
\subsection{プロキシ}
\subsection{ゲートウェイ}

\section{Hypertext Transfer Protocol}

\section{関連研究}
