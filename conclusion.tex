\documentclass[12pt,a4paper]{jbook}
\usepackage{mm-thesis}
\usepackage[dvipdfmx]{graphicx}
\usepackage{cite}
\usepackage{comment}
\usepackage{docmute}
\usepackage{color}
\usepackage{moreverb}
\usepackage{listings}
\usepackage{ascmac}
%\usepackage{amsmath}
%\usepackage{amsthm}
%\usepackage{amsfonts}

\lstset{
	%枠外での自動改行
 	breaklines = true,
 	%標準の書体
 	basicstyle = {\small},
 	%枠 "t"は上に線を記載, "T"は上に二重線を記載
	%他オプション:leftline,topline,bottomline,lines,single,shadowbox
 	frame = TB,
 	%タブの大きさ
 	tabsize = 2,
 	%キャプションの場所("tb"ならば上下両方に記載)
 	captionpos = t,
 	%行番号の位置
 	numbers = left,
 	%自動改行後のインデント量(デフォルトでは20[pt])	
 	breakindent = 30pt,
	%左右の位置調整 	
 	xleftmargin=30pt,
 	xrightmargin=30pt,
	%プログラム言語(複数の言語に対応,C,C++も可)
 	%language = Python, 	
 	%背景色と透過度
 	%backgroundcolor={\color[gray]{.90}},
 	%コメントの書体
 	%commentstyle = {\itshape \color[cmyk]{1,0.4,1,0}},
 	%関数名等の色の設定
 	%classoffset = 0,
 	%キーワード(int, ifなど)の書体
 	%keywordstyle = {\bfseries \color[cmyk]{0,1,0,0}},
 	%表示する文字の書体
 	%stringstyle = {\ttfamily \color[rgb]{0,0,1}},
 	%frameまでの間隔(行番号とプログラムの間)
 	%framesep = 5pt,
 	%行番号の間隔
 	%stepnumber = 1,
	%行番号の書体
 	%numberstyle = \tiny,
}
\renewcommand{\lstlistingname}{Code}
\begin{document}
\newpage

\color{red}
\chapter{おわりに}
本研究では、ウェブの安全性解析に形式手法を用いることを目的とし、ウェブを構成する要素の状態変化を表現するための時相論理を、Alloy上で表現するための記述法を考案した。
また、この記述法を用いてキャッシュを表現したウェブセキュリティモデルを実装し、その表現能力を評価した。

提案記述法では、様々なウェブの構成要素の状態を表現可能な汎用的なクラスを定義した。
また、このクラスのインスタンスを時系列順に並び替えるために利用できる二つの述語を実装した。
これにより、ウェブに含まれる要素の時系列順の状態変化を表現することが可能となる。

また、この記述法を用いてキャッシュを実装したウェブセキュリティモデルでは、実際にキャッシュの基本的な動作を表現できることを確認した。
これに加えて、既存のウェブセキュリティモデルでは表現できなかった状態遷移を含む四つの攻撃法について表現可能であることを確認し、モデルの表現能力の向上を達成した。
本研究では、これらの攻撃法のモデルによる表現を実現したが、対策法の考案には至っていない。
ここで得られた出力結果を元に、攻撃法が成功しない条件の発見が今後の課題である。

また一方で、本研究で実装した提案モデルではHTTPSの拡張プロトコルであるHTTP Strict Transport Security\cite{hsts}、Public Key Pinning Extension for HTTP\cite{hpkp}に対する応用も課題である。
これらは現在ウェブでの実用化がすすめられているプロトコルであり、安全性解析を要するものであるが、プロトコルの操作に必要なヘッダが提案モデルに不足しているため、提案モデルでの安全性解析が不可能である。
しかし、これらのプロトコルがキャッシュを用いてHTTPSの安全性を向上させたものであるため、提案モデルの拡張によるこれらのプロトコルの表現は現実的であると考えている。
\color{black}

\end{document}
