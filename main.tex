\documentclass[12pt,a4paper]{jbook}

%\usepackage{mm-thesis}
\usepackage[dvipdfmx]{graphicx}
\usepackage{cite}
\usepackage{comment}
\usepackage{docmute}
\usepackage{color}
%\usepackage{amsmath}
%\usepackage{amsthm}
%\usepackage{amsfonts}

\usepackage{mm-thesis}
\usepackage[dvipdfmx]{graphicx}
\usepackage{latexsym}
\usepackage{amsmath}
\usepackage{amsthm}
\usepackage{amsfonts}
\usepackage{cite}
\usepackage{comment}
%\usepackage{docmute}

\thesis{\master}
\title{webの汎用的解析に向けた時相論理の表現による形式検証}
\author{島本 隼人}
\supervisor{藤原 融 教授}
\deadline{2017年2月15日}

\abstract{
こうがい
}

\keyword{きーわーど1, きーわーど2}

\begin{document}

\coverpage
\tableofcontents
\listoffigures
\listoftables
\body

\chapter{はじめに}

\chapter{準備}
\section{形式検証}
\subsection{Alloy}
\section{セキュリティモデル}
\section{時相論理}
\section{キャッシュ}
\subsection{Browser Cache Poisoning攻撃}
\section{中継者}
\section{Hypertext Transfer Protocol}
\section{関連研究}

\chapter{既存のモデル}
\section{拡張を想定した汎用モデル}
\section{Cookieを包括するモデル}
\section{問題点}

\chapter{時相論理を包括するモデル}
\section{時相論理の表現方法}
\section{時相論理の応用}
\subsection{キャッシュ}
\subsection{中継者}

\chapter{事例研究}
\section{キャッシュの基本動作}
\section{中継者の基本動作}
\section{BCP攻撃}

\chapter{おわりに}

\acknowledgement
まず,本研究を進める全過程において多大な御指導を賜りました藤原融教授に深く感謝致します.
また,矢内直人助教,岡村真吾招聘准教授に心より御礼申し上げます.
そして,日常の議論を通じて多くのアドバイスや指摘を下さった藤原研究室の皆様ならびに,円滑な研究生活のために諸事務手続きを行って下さった秘書の方々に感謝致します.

\bibliographystyle{junsrt}
\bibliography{list} 

\end{document}


