\documentclass[a4paper]{jarticle}

%\input{macro}

\topmargin -28mm
\oddsidemargin -15mm
\evensidemargin -15mm
\textwidth 185mm
\textheight 275mm
\columnsep 6mm

\def\zenjitu{平成30年2月15日}
\def\toujitu{平成30年2月16日}
\def\zenjituyoubi{{\zenjitu}(木)}
\def\toujituyoubi{{\toujitu}(金)}
\def\busuu{26}
\def\room{情報科学研究科B棟1階B101}
% \Eps{filename}{caption}{label}{scale}
%\newcommand{\Eps}[4]{ \begin{figure}[hbtp] \begin{center} \psbox[scale=#4]{#1} \end{center} \vspace{-8mm} \caption{#2} \label{#3} \end{figure} }

\makeatletter
\def\section{\@startsection{section}{2}{\z@}{.8ex plus .8ex minus
.2ex}{.05ex plus .07ex}{\large\bf}}
\makeatother
\makeatletter
\def\subsection{\@startsection{subsection}{2}{\z@}{.8ex plus .8ex minus
.2ex}{.05ex plus .07ex}{\bf}}
\makeatother


\pagestyle{empty}

\begin{document}

\baselineskip 4.75mm
%行間の指定をここで行う

\twocolumn
[
\footnotesize 
\begin{center}
{大阪大学大学院情報科学研究科 マルチメディア工学専攻 博士前期課程修士学位論文発表会資料
\hfill \toujitu}\\
\medskip
{\large
{\bf Alloyにおける時相論理の記述法とWebの安全性解析手法への応用}\\
}
\medskip
{\large
        島本 隼人(セキュリティ工学講座)
}
\end{center}
]

\section{はじめに}
近年, 製造, 流通, 医療, 教育, エンターテイメント, ネットワーク分野を始めとするさまざまな分野におけるディジタル化技術の急速な発展により, 複数のメディアが統合された高度マルチメディア情報処理環境の構築に対する要望が強い. 先進的なマルチメディア情報処理環境を構築するには, その核技術として, マルチメディアコンテンツを獲得・編集・構造化し, 大規模なマルチメディアシステムを構築する方法論の確立が急務である. 特に, 真に有効なマルチメディアシステムを構築するには, 高速な情報通信ネットワークを介してマルチメディアコンテンツを高速に加工・編集・蓄積するためのメディア情報処理技術が重要な鍵となる.  

このためには, 時空間的な側面とコンテンツ構造の進化性を中心に据えたマルチメディアコンテンツの編集・構造化の高速化, マルチメディアデータベースの構築・管理技法, コンテンツのアクセス権管理・版権管理・配送管理等を中心とする堅牢なセキュリティ技術に基づくコンテンツアクセスアーキテクチャ, マルチメディアを駆使した電子商取引システムやサプライチェーン管理システムなどのビジネス情報システムの開発技術, さらにはマルチメディアコンテンツをもとに生成される仮想現実や拡張現実空間での操作体系を含めた, 高度なヒューマンインタフェース技術の確立が必須となる. そこで, 本専攻ではこれらの多様なユーザからの要求に応じることができる学問体系の確立を目指す.  

\section{修士学位論文の意義}
修士学位論文は, 大学院博士前期課程における研究の成果をまとめて作成する
ものであり, 大学院の教育課程2年間の締めくくりとしての重要な意義を持っ
ている. 本学位論文は, 単なる実験の報告書とは異なり, 諸君の執筆した一つ
の文献として本専攻に保存され, 教員や学生はじめ一般の閲覧に供せられる. 
そのため, 内容はいうまでもなく, 論文の体裁や, 構成, 文章, 語句等につい
ても細心の注意を払う必要がある. 研究に関しては, この時点までに十分な成
果が上げられていることと思うが, 修士学位論文の作成は, 「研究成果をいか
に論文として表現するか」という技術を身につける訓練でもあることを念頭に
おいて, 執筆に励んで欲しい. 

また, 修士学位論文の発表会は取り組んできた研究について発表することによ
り, 学位を授与するに適当な成果が得られているかどうかを判定する試験でもある. 
そのため, 本発表会についても内容のみならず, 発表の形式
やプレゼンテーション資料の書き方等, 十分に留意されたい. 

\vspace{-0.2cm}
\section{修士学位論文に関する提出物}
修士学位論文と発表会資料を, 以下の要領でそれぞれ期限までに提出すること.
なお, 英文アブストラクトは, 当専攻では提出の必要はない. 

\bigskip
\noindent
{\bf 修士学位論文}
\begin{itemize}
\item 修士学位論文は, {\zenjituyoubi}正午までに, 指導教授に提出すること. 
\item 修士学位論文の言語は日本語または英語とする. 
\item 印刷には A4 の用紙を用いること. 
\item 枚数は特に制限しないが, 付録を除いて, 
システム系論文の場合は 50 ページを目安とし, 理論系論文の場合は 40 ページを目安
とする. 自分の論文がどちらにあたるかは, 指導教授に確認しておくこと.
\item 1 ページあたりの字数, 行数は, 日本語であれば 40字×35行とする. 
また, 文字の大きさは12ptとする. 英語の場合も行数は同じで,
文字の大きさは12ptとする.
ページ内における印字領域の大きさは日本語の場合とほぼ同じとする.
\end{itemize}

\bigskip
\noindent
{\bf 発表会資料}
\begin{itemize}
\item 修士学位論文発表会の際に, 論文の内容をA4用紙1枚にまとめた「発表会資料」を準備する. 
\item 発表会資料の言語は日本語または英語とする. 
\item 用紙左上に「大阪大学大学院情報科学研究科 マルチメディア工学専攻 博士前期課程修士学位論文発表会資料」, 右上に「{\toujitu}」と記入すること. 
\item 題目, 氏名, 所属講座名を記入すること. 
\item 内容は, 例えば次のような構成でまとめること. 但し, 研究の性格によっては, 必ずしもこの構成が適切とは限らないので, 聴講者に分かりやすくなる
 ように柔軟に考えること.\\
 (a) 目的\\
 (b) 研究内容/方法\\
 (c) 成果/結果/関連発表論文
\item 上記資料を PDF ファイルとして準備し, {\zenjituyoubi}正午までに, 専攻長が指示するメールアドレス宛に送付すること. PDFファイルのファイル名は「学籍番号-名前(研究室名).pdf」とすること. また, {\busuu}部印刷し, 専攻長まで提出すること. 
なお,メールでの送付および印刷物での提出は,研究室で
まとめて行うこと (特段の事情がある場合はこの限りではない).
\end{itemize}

\section{修士学位論文発表会}
発表会は, {\toujituyoubi}に, 吹田キャンパス{\room}において行なう. 
研究棟外の方のための待機場所を, A308に準備する予定である. 
プログラム等詳細については追って連絡する. 

各人への割当時間は, 発表時間9分, 質疑応答6分とする. 

発表のメディアはプロジェクタを利用した PC でのプレゼンテーションとする. 
他に特別な機器を利用する予定の者は事前に専攻長に連絡し許可を得ること. 

発表スライドの一枚目には, 題目, 発表者名, 所属講座名を記したものを準備すること. 

\section{おわりに}
本サンプルでは, 大阪大学大学院情報科学研究科マルチメディア工学専攻における修士学位論文発表会資料について述べた. 


\end{document}
