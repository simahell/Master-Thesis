\documentclass[12pt,a4paper]{jbook}
\usepackage{mm-thesis}
\usepackage[dvipdfmx]{graphicx}
\usepackage{cite}
\usepackage{comment}
\usepackage{docmute}
\usepackage{color}
%\usepackage{amsmath}
%\usepackage{amsthm}
%\usepackage{amsfonts}

\begin{document}

ウェブの発展に伴い個人情報を取り扱いが増加しているため、その通信の安全性が重要となっている。
一方で、ウェブの構造は一般的なシステムと比べ複雑となっている。
このようなシステムに対しては、適当な入力を与えて実際の動作を確認するといった従来の手法による安全性解析は時間がかかり、また、漏れが生じる可能性が高く現実的ではない。
そこで近年は、システムの仕様と安全性要件を抽象化したセキュリティモデルを用いた形式手法による解析が盛んにおこなわれている。
形式手法を用いることで、漏れのない自動化した検査を実現できる。
しかし、形式手法の入力となるセキュリティモデルの記述が不十分である場合、解析結果に潜在的な危険性が存在する場合がある。
このような背景から、セキュリティモデルの検討が重要である。

本研究では、既存のウェブセキュリティモデルにおいて包括されている時相論理が不十分であることに注目する。
セキュリティモデルにおいて、時相論理はシステムの状態変化を表現する能力を持ち、システムの正確な安全性解析には不可欠な項目である。
しかし、現状の既存のモデルでの時相論理では、ある一つの通信によるウェブの状態変化しか捉えることができず、二つ以上の通信が互いに及ぼす影響性を考えることができない。
これは、捉えることのできる状態の変化が対となるリクエストとレスポンス間に限られているためである。
この理由としては、既存のウェブセキュリティモデルで利用しているAlloyという形式手法ツールが時相論理に元々対応していないことが挙げられる。
時相論理では「次の状態では~」といった時系列順を用いた論理が用いられるため、Alloyを利用する上ではその時系列を捉えられるよう独自の記述法が必要となる。
\color{red}
%既存文:しかし、グラフを用いた直観的な出力結果を得られる点や、拡張を前提とした基礎研究が確立されており、これを利用した後続の研究も進められている点を考え、Alloyを用いたモデルの拡張が求められる。
%comment:akhaweのモデルを拡張することの信頼性について掘り下げる
しかし、ウェブに対して形式手法を用いる研究ではAlloyを用いた先行研究が確立されており、かつ、その先行研究を応用した研究も数多く存在するため、その先行研究の信頼性は確かなものである。
これに加え、Alloyを使用することでグラフを用いた直感的な出力結果を得ることができ、安全性解析をより効率的に進めることができるため、Alloyを用いたモデルの既存モデルの拡張が求められる。
\color{black}
このような背景から、様々なウェブ要素に対して汎用性のある、Alloyでの時相論理の記述法を作成する。
この時相論理を実現する方針は、対象の要素の各状態を保存するインスタンスのフォーマットを定め、その時系列を考えるために必要となる「最初の状態」と「ある状態の次状態」を判定する述語を作成するというものである。
この述語を利用することで状態の時系列変化を捉えられるため、前もって対象のウェブの要素が時間変化によってどのような状態変化が発生しうるかを仕様書から明らかにしておき、時系列順で前後に並ぶ状態間に対する論理式として記述することで時相論理を実現できる。

また上記の時相論理の設計に加え、この記述法を用いてキャッシュを記述した、新たなウェブセキュリティモデルを作成し評価を行う。
\\(ここから事例研究について書きます)

\end{document}
