\documentclass[12pt,a4paper]{jbook}
\usepackage{mm-thesis}
\usepackage[dvipdfmx]{graphicx}
\usepackage{cite}
\usepackage{comment}
\usepackage{docmute}
\usepackage{color}
%\usepackage{amsmath}
%\usepackage{amsthm}
%\usepackage{amsfonts}

\begin{document}

ウェブの発展に伴いその通信の安全性が重要となっている。
一方で、ウェブの構造は一般的なシステムと比べ複雑となっており、適当な入力を与えて実際の動作を確認するといった従来の安全性解析手法は漏れが生じる可能性が高い。
そこで近年は、システムの仕様と安全性要件を抽象化したセキュリティモデルを用いた形式手法による解析が注目されている。
セキュリティモデルが必要項目を正確に記述できている場合、漏れのない解析結果を得ることができる。
したがって、形式手法を用いて漏れのない安全性解析を行うためには、セキュリティモデルが十分にシステムを包括していることが重要である。

%edit:第二段落では研究における問題点
本研究では、既存のウェブのセキュリティモデルにおいて包括されている時相論理が不十分であることに注目する。
一般に、時相論理はシステムの状態変化を表現する能力を持つことから、システムの正確な安全性解析には不可欠な項目である。
現状の既存モデルの時相論理では、ある一つの通信によるウェブの状態変化しか捉えることができず、二つ以上の通信が互いに及ぼす影響を捉えていない。
一方で、この既存モデルを応用した研究も多く存在するため、この既存モデルを基礎として研究を行うことが本研究分野での貢献につながる。
このような背景から、既存モデルの時相論理の拡張が求められる。

%edit:第三段落では問題点に対する解決法
しかし、そもそも既存モデルの実装に用いられているツールであるAlloy Analyzerは時相論理を表現する機能を持たない。
これが原因となり、既存モデルの時相論理に強い制限がかけられている。
一方で、Alloy Analyzerは出力結果の表示形式に直感的に把握できるグラフ形式を利用でき、これは要素数が一般的なシステムに比べて多いウェブの円滑な安全性解析を可能にする。
したがって、拡張した時相論理を実装するためのAlloy上での記述法を提案する。
この記述法により、各時点での状態を表すためのクラスをあらかじめ指定されているフォーマットに従って作成し、専用の述語を用いることで時系列に沿った状態変化を捉えることが可能になる。

また、この提案記述法を用いてキャッシュを表現し、基礎モデルの拡張を行う。
\\(ここから事例研究について書きます)

\end{document}
