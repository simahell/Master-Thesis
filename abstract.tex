\documentclass[12pt,a4paper]{jbook}
\usepackage{mm-thesis}
\usepackage[dvipdfmx]{graphicx}
\usepackage{cite}
\usepackage{comment}
\usepackage{docmute}
\usepackage{color}
\usepackage{moreverb}
\usepackage{listings}
\usepackage{ascmac}
%\usepackage{amsmath}
%\usepackage{amsthm}
%\usepackage{amsfonts}

\lstset{
	%枠外での自動改行
 	breaklines = true,
 	%標準の書体
 	basicstyle = {\small},
 	%枠 "t"は上に線を記載, "T"は上に二重線を記載
	%他オプション:leftline,topline,bottomline,lines,single,shadowbox
 	frame = TB,
 	%タブの大きさ
 	tabsize = 2,
 	%キャプションの場所("tb"ならば上下両方に記載)
 	captionpos = t,
 	%行番号の位置
 	numbers = left,
 	%自動改行後のインデント量(デフォルトでは20[pt])	
 	breakindent = 30pt,
	%左右の位置調整 	
 	xleftmargin=30pt,
 	xrightmargin=30pt,
	%プログラム言語(複数の言語に対応,C,C++も可)
 	%language = Python, 	
 	%背景色と透過度
 	%backgroundcolor={\color[gray]{.90}},
 	%コメントの書体
 	%commentstyle = {\itshape \color[cmyk]{1,0.4,1,0}},
 	%関数名等の色の設定
 	%classoffset = 0,
 	%キーワード(int, ifなど)の書体
 	%keywordstyle = {\bfseries \color[cmyk]{0,1,0,0}},
 	%表示する文字の書体
 	%stringstyle = {\ttfamily \color[rgb]{0,0,1}},
 	%frameまでの間隔(行番号とプログラムの間)
 	%framesep = 5pt,
 	%行番号の間隔
 	%stepnumber = 1,
	%行番号の書体
 	%numberstyle = \tiny,
}
\renewcommand{\lstlistingname}{Code}
\begin{document}

\color{red}
ウェブの発展に伴い個人情報を取り扱いが増加しているため、その通信の安全性が重要となっている。
一方で、ウェブの構造は一般的なシステムと比べ複雑であり従来の手法による安全性解析は現実的ではないため、システムの仕様と安全性要件を抽象化したセキュリティモデルを用いた形式手法による解析が近年盛んにおこなわれている。
しかし、形式手法の入力となるセキュリティモデルの記述が不十分である場合、解析結果に潜在的な危険性が存在する場合がある。
このような背景から、セキュリティモデルの検討が重要である。

本研究では、既存のウェブセキュリティモデルにおいて包括されている時相論理が不十分であることに注目する。
セキュリティモデルにおいて、時相論理はシステムの状態変化を表現する能力を持ち、システムの正確な安全性解析には不可欠な項目である。
しかし、既存のモデルでの時相論理では、ある一つの通信によるウェブの状態変化しか捉えることができず、二つ以上の通信が互いに及ぼす影響性を考えることができない。
これは、捉えることのできる状態の変化が対となるリクエストとレスポンス間に限られているためである。
このような背景から、様々なウェブ要素に対して汎用性のある、全体を通して解析可能な時相論理の記述法を作成する。
この時相論理の設計の方針は、対象のウェブ要素の状態変化が発生する各タイミングにおける状態を保存するインスタンスを作成し、それらを時系列順に並べた際の前後のインスタンス間に生じる条件を記述することで、システム全体を通じてのその要素の状態変化を捉えるというものである。
そこで、この設計を実現するため、状態を保存するインスタンスのフォーマットを定義し、それを利用している場合に時系列順の並び替えを容易にする述語を提供する。
この時相論理を利用する際には、指定のフォーマットに従って記述し、時系列順の前後の関係性を記述することで時相論理を導入することができる。

また上記の時相論理の設計に加え、この記述法を用いてキャッシュや中継者を記述した、新たなウェブセキュリティモデルを作成し評価を行う。
\\(ここから事例研究について書きます)


\color{black}

\end{document}
