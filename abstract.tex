\documentclass[12pt,a4paper]{jbook}
\usepackage{mm-thesis}
\usepackage[dvipdfmx]{graphicx}
\usepackage{cite}
\usepackage{comment}
\usepackage{docmute}
\usepackage{color}
%\usepackage{amsmath}
%\usepackage{amsthm}
%\usepackage{amsfonts}

\begin{document}

ウェブの発展に伴いその通信の安全性が重要となっている.
一方で,ウェブの構造は一般的なシステムと比べ複雑となっており,適当な入力を与えて実際の動作を確認するといった従来の安全性解析手法は漏れが生じる可能性が高い.
そこで近年は,システムの仕様と安全性要件を抽象化したセキュリティモデルを用いた形式手法による解析が注目されている.
セキュリティモデルが必要項目を正確に記述できている場合,漏れのない解析結果を得ることができる.
したがって,形式手法を用いて漏れのない安全性解析を行うためには,セキュリティモデルが十分にシステムを包括していることが重要である.

一般に,時相論理はシステムの状態変化を表現する能力を持つことから,システムの正確な安全性解析には不可欠である.
しかし,既存のウェブのセキュリティモデルの時相論理では,ある1つの通信間でのウェブの状態変化しか捉えることができず,2つ以上の通信が互いに及ぼす影響を捉えていない.
これは実装に用いられているモデル検証ツールAlloy Analyzerが時相論理を表現する機能を持たず,拡張した時相論理の実装が容易ではないことに起因する.
しかし,この既存モデルの実装コードを利用した研究も多く存在するため,これを可能な限り引き継ぐことでモデル実装の信頼性を高めることができる.

Alloyにおけるウェブの構成要素の状態変化を表現する時相論理を実装するため,これを可能にするAlloy上での記述法を提案する.
まず,各時点での状態を表現するためのクラスを作成する.
次に,このクラスを時系列順を把握するための述語を作成することで,状態遷移を表現可能とする.
ここで述語の記述が容易になるよう,クラスのフォーマットを定義し,状態クラスを既存モデル内の適切な位置に追加する.
これらにより専用の状態クラスの形式に従って記述し,ウェブの動作仕様から各状態間で発生しうる変化を整理することで,ウェブの時系列に沿った状態変化を表現可能となる.
また,この提案記述法を用いてキャッシュを実装したモデルに対して,いくつかの既知の攻撃例を用いてモデルの表現能力の向上を確認し,提案記述法の有用性を示す.
\end{document}
