\documentclass[12pt,a4paper]{jbook}
\usepackage{mm-thesis}
\usepackage[dvipdfmx]{graphicx}
\usepackage{cite}
\usepackage{comment}
\usepackage{docmute}
\usepackage{color}
%\usepackage{amsmath}
%\usepackage{amsthm}
%\usepackage{amsfonts}

\begin{document}
\newpage

\chapter{はじめに}
\label{sec:introduction}
近年,個人情報をウェブ上でやり取りする機会が増えている.
これは,ウェブを利用できる携帯端末が普及したことにより,ウェブ上のサービスが充実してきたことに起因する.
例えば,以前では窓口やATMによって振り込みの手続きを行っていたが,現在ではネットバンキングを利用することでどこからでも簡単に手続きが可能である.
このような個人情報を含む通信の増加により,ウェブの安全性が極めて重要となっている.
一方で,システムの安全性を検査する際には,システムに適当な入力を与え動作をシミュレーションする方法が一般的に採用される.
しかし,ウェブの構造は非常に複雑である.
これは,ウェブの運用開始からこれまでにあらゆる拡張が成され,多様な要素が存在するためである.
このように複雑なウェブというシステムに対して,シミュレーションでは検査に漏れが生じる可能性が多く,安全性を確実に保証することはできない.

本研究では,ウェブというシステムの安全性を証明する手法として形式手法を利用する.
この手法の利用にはまず,想定される攻撃者の能力,システムの運用環境,満たされるべき安全性要件を論理式を用いて表現したセキュリティモデルを用意する.
形式手法はこのセキュリティモデルを利用し,システムが安全性を満たしているかを数学的に検査する.
したがって,検査対象のシステムの仕様を正確に表現したセキュリティモデルを用いることで,漏れのない精密な検査が可能となる.

しかし,前述の通り,ウェブというシステムには様々な要素が含まれており,複雑な構造となっている.
昨今のウェブの目まぐるしい発展により,新たな技術が次々に追加されているためである.
このような状況を受けて,これらの膨大な要素の中から現在広く使われている要素を抽出し,それらの要素に注目したウェブセキュリティモデルが提案されている\cite{based-model,cookie-model}.
しかし,これらの既存モデルにおける時相論理の表現能力は,ある1つの通信のやり取りの前後における状態変化にのみ対応しており,ある通信が他の通信に及ぼす影響を捉えることができない.
そもそも,時相論理はシステムの各状態間の関係性を表し,また,ウェブは時間ごとに状態が刻々と変化するシステムであるため,ウェブセキュリティモデルに対する時相論理の導入は不可欠である.
本来,ウェブにおいては複数の通信が並列していることが自然であることから,この既存モデルの時相論理の表現力は不十分である.
しかしながら,既存モデル\cite{based-model}はウェブに対する形式手法の基礎研究として確立されており,またこれを基にしたクッキーの研究\cite{cookie-model}を始めとし,その他の様々な研究\cite{chaitanya2017formal, bagheri2016practical, chen2015aspire, nelson2013aluminum, somorovsky2011all}が進められているため,これらのモデルを基に拡張を行うことが重要である.

また,このような強い制限が既存モデルの時相論理にかけられている要因は,それらのモデルがAlloy Analyzerという形式手法ツールを用いて実装されていることである.
Alloy Analyzerには時相論理を援助する機能を持たないため,ツールに頼らない独自の記述法が必要となる.
また,Alloy Analyzerは出力結果をグラフを用いた形式で出力でき,この出力形式が構造が複雑なウェブの安全性解析に適している.
このような背景から本研究の目的は,Alloy Analyzer上で拡張した時相論理を表現できる記述法を作成することである.

また,本研究では上記の時相論理の記述法の作成に加えて,時相論理の実装により表現できる要素としてキャッシュに注目し,その実装を行う.
キャッシュは広く一般に使われているウェブの構成要素である一方で,既存モデルに含まれていない要素である.
また,実際にキャッシュを用いた攻撃法\cite{bcpattack}も報告されており,キャッシュはウェブの安全性を評価する上で欠かすことのできない概念である.
本研究では,提案する時相論理の記述法を用いてキャッシュを実装したウェブセキュリティモデルも提案する.

本論文の構成は以下の通りである.
2章では,形式手法とウェブに関連する知識を述べる.
3章では,本研究の基礎となる既存のウェブセキュリティモデルとその問題点に述べる.
4章では,Alloyで利用可能な時相論理の記述法を提案する.
5章では,本研究で提案する時相論理の記述法を利用した,キャッシュを包括するウェブセキュリティモデルを提案する.
6章では,現実に起こりうる事例を取り上げ検証結果を確認することで,提案モデルを評価する.
\end{document}
