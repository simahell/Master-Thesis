\documentclass[12pt,a4paper]{jbook}
\usepackage{mm-thesis}
\usepackage[dvipdfmx]{graphicx}
\usepackage{cite}
\usepackage{comment}
\usepackage{docmute}
\usepackage{color}
\usepackage{moreverb}
\usepackage{listings}
\usepackage{ascmac}
%\usepackage{amsmath}
%\usepackage{amsthm}
%\usepackage{amsfonts}

\lstset{
	%枠外での自動改行
 	breaklines = true,
 	%標準の書体
 	basicstyle = {\small},
 	%枠 "t"は上に線を記載, "T"は上に二重線を記載
	%他オプション:leftline,topline,bottomline,lines,single,shadowbox
 	frame = TB,
 	%タブの大きさ
 	tabsize = 2,
 	%キャプションの場所("tb"ならば上下両方に記載)
 	captionpos = t,
 	%行番号の位置
 	numbers = left,
 	%自動改行後のインデント量(デフォルトでは20[pt])	
 	breakindent = 30pt,
	%左右の位置調整 	
 	xleftmargin=30pt,
 	xrightmargin=30pt,
	%プログラム言語(複数の言語に対応,C,C++も可)
 	%language = Python, 	
 	%背景色と透過度
 	%backgroundcolor={\color[gray]{.90}},
 	%コメントの書体
 	%commentstyle = {\itshape \color[cmyk]{1,0.4,1,0}},
 	%関数名等の色の設定
 	%classoffset = 0,
 	%キーワード(int, ifなど)の書体
 	%keywordstyle = {\bfseries \color[cmyk]{0,1,0,0}},
 	%表示する文字の書体
 	%stringstyle = {\ttfamily \color[rgb]{0,0,1}},
 	%frameまでの間隔(行番号とプログラムの間)
 	%framesep = 5pt,
 	%行番号の間隔
 	%stepnumber = 1,
	%行番号の書体
 	%numberstyle = \tiny,
}
\renewcommand{\lstlistingname}{Code}
\begin{document}

\chapter{はじめに}
近年、個人情報をウェブ上でやり取りする機会が増えている。
これは、ウェブを利用できる携帯端末が普及したことにより、ウェブ上のサービスが充実してきたことに起因する。
例えば、以前では窓口やATMによって振り込みの手続きを行っていたが、現在ではネットバンキングを利用することでどこからでも簡単に手続きを行うことができる。
したがって、通信内容に含まれる個人情報の増加によりウェブの安全性が極めて重要となっている。
一方で、システムの安全性を検査する際には、システムに適当な入力を与え動作をシミュレーションする方法が一般的に採用される。
しかし、企業等で運用されるシステムに比べウェブの構造は非常に複雑である。
これは、ウェブの運用開始からこれまでにあらゆる拡張が成され、多様な要素が存在するためである。
このように複雑なウェブに対して、適当な入力を与えるシミュレーションでは検査に漏れが生じる可能性が多く、安全性を確実に保証することはできない。

本研究では、ウェブというシステムの安全性を証明する手法として形式手法を利用する。
この手法の利用にはまず、想定される攻撃者の能力、システムの運用環境、満たされるべき安全性要件を論理式を用いて表現したセキュリティモデルを用意する。
形式手法はこのセキュリティモデルを利用し、システムが安全性を満たしているかを数学的に検査する。
これにより漏れのない、より精密な検査が可能になる。
また、この手法の長所として、セキュリティモデルを入力とし専用の解析ツールを用いることで解析を自動化でき、労力を大幅に削減できることが挙げられる。

しかし、前述の通り、ウェブというシステムには様々な要素が含まれており、複雑な構造となっている。
これは、ウェブの目まぐるしい発展により、従来の技術よりも優れた技術が次々に追加されているためである。
このような状況を受けて、これらの膨大な要素の中から現在広く使われている要素を抽出し、それらの要素に注目したウェブセキュリティモデルが提案されている\cite{based-model,cookie-model}。
しかし、これらの既存モデルにおける時相論理が表現できることは、ある一つの通信のやり取りの前後における状態変化のみであり、ある通信が他の通信に及ぼす影響を考えることができない。
本来、ウェブにおいては複数の通信が並列していることが当然であり、この既存モデルでの時相論理の表現力では不十分である。
そもそも、時相論理はシステムの各状態間の関係性を表し、ウェブは時間ごとに状態が刻々と変化するシステムであるため、時相論理による表現は不可欠である。
しかしながら、これらの既存モデル\cite{based-model,cookie-model}は、Alloy Analyzerという形式手法ツールを用いて実装されており、このツールは時間軸を表現するといった機能を持たない。
そのためAlloy上で時相論理を表現するには、ツールの機能に頼らない時相論理の記述法が必要となる。
一方で、既存モデル\cite{based-model}はウェブに対する形式手法の基礎研究として確立されており、またこれを基にしたCookieの研究\cite{cookie-model}を始めとする様々な研究
\color{red}
\cite{chaitanya2017formal, bagheri2016practical, chen2015aspire, nelson2013aluminum, somorovsky2011all}
\color{black}
が進められているため、これらのモデルを基に拡張を行うことが重要である。
このような背景から本研究の目的は、Alloyを用いて、より正確な安全性検証のために時相論理の拡張を行ったウェブセキュリティモデルの提案である。

また、本研究では上記の時相論理の表現を加えて、時相論理の実装により表現できる要素としてキャッシュに注目している。
キャッシュは広く一般に使われているウェブの構成要素である一方で、既存モデルに含まれていない要素である。
しかし、実際にこれらの要素は近年の成果として報告されている攻撃法\cite{bcpattack}にも利用されており、ウェブの安全性を評価する上で欠かすことのできない概念である。
本研究では、拡張した時相論理を用いてキャッシュの実装も行う。

本論文の構成は以下の通りである。
二章では、形式手法とウェブに関連する知識を述べる。
三章では、本研究の基礎となる既存のウェブセキュリティモデルとその問題点に述べる。
四章では、時相論理を包括したウェブセキュリティモデルを提案する。
五章では、現実に起こりうる事例を取り上げ検証結果を確認することで、提案モデルを評価する。
\end{document}
