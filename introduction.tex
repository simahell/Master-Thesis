\documentclass[12pt,a4paper]{jbook}
\usepackage{mm-thesis}
\usepackage[dvipdfmx]{graphicx}
\usepackage{cite}
\usepackage{comment}
\usepackage{docmute}
\usepackage{color}
%\usepackage{amsmath}
%\usepackage{amsthm}
%\usepackage{amsfonts}

\begin{document}

\chapter{はじめに}
近年、個人情報をウェブ上でやり取りする機会が増えている。
これは、ウェブを利用できる携帯端末が普及したことにより、ウェブ上のサービスが充実してきたことに起因する。
例えば、以前では窓口やATMによって振り込みの手続きを行っていたが、現在ではネットバンキングを利用することでどこからでも簡単に手続きを行うことができる。
一方で、通信内容に含まれる個人情報の増加によりウェブの安全性が極めて重要となっているが、その安全性の解析は一般に難しい。
これは、ウェブの構造が複雑かつ多種多様な要素を含んでいるためである。
\textcolor{red}{
つまり、ウェブは現在の社会にとって不可欠であり、かつ、機密性の高い情報を扱うことも多く、不具合のない確かな安全性が求められるシステムであるが、その安全性の証明に一般的な手法を用いると膨大な手間がかかるため現実的ではない。
}
%このようなシステムでは、安全性の証明のために、発生しうる可能性全てを考慮することは膨大な労力がかかるため現実的ではない。

\textcolor{red}{
本研究では、ウェブというシステムの安全性を証明する手法として形式手法を利用する。
この手法の利用にはまず、想定される攻撃者の能力、システムの運用環境、満たされるべき安全性要件を論理式を用いて表現したセキュリティモデルを用意する。
形式手法はこのセキュリティモデルを利用し、システムが安全性を満たしているかを数学的に検査する。
これにより証明漏れのない、より精密な検査が可能になる。
また、この手法の長所として、セキュリティモデルを入力とし専用の解析ツールを用いることで解析を自動化でき、労力を大幅に削減できることが挙げられる。
}
%本研究では、あるシステムが安全性を満たしているかを判断する手法として、セキュリティモデルを利用する安全性解析に注目する。
%セキュリティモデルでは想定される攻撃者の能力、システムを運用する環境、満たされるべき安全性要件を、様々なパラメータを数値化することで定義されている。
%このセキュリティモデルを考察することで、検証対象のシステムが安全性を満たしているかを定量的に評価することが可能になる。
%この手法の長所として、セキュリティモデルを入力とし専用の解析ツールを用いることで解析を自動化でき、労力を大幅に削減できることが挙げられる。

しかし、前述の通り、ウェブというシステムには様々な要素が含まれており、複雑な構造となっている。
これは、ウェブの目まぐるしい発展により、従来の技術よりも優れた技術が次々に追加されているためである。
このような状況を受けて、これらの膨大な要素の中から現在広く使われている要素を抽出し、それらの要素に注目したウェブセキュリティモデルが提案されている\cite{webmodel,cookie-model}。
\textcolor{red}{
しかし、これらの既存モデルにおける時相論理の表現できることは、ある一つの通信のやり取りの前後における状態変化のみであり、ある通信が他の通信に及ぼす影響を考えることができない。
ウェブでは複数の通信が並列していることが当然であり、この既存モデルでの時相論理の表現力では不十分である。
%しかし、これらの既存モデルにおける時相論理の表現は不十分である。
そもそも、時相論理はシステムの各状態間の関係性を表し、ウェブは時間ごとに状態が刻々と変化するシステムであるため時相論理による表現は不可欠である。
}
このような背景から本研究の目的は、より正確な安全性検証のために拡張した時相論理をもつウェブセキュリティモデルの提案である。

また、本研究では上記の時相論理の表現を加えて、時相論理の実装により表現できる要素としてキャッシュと中継者に注目している。
これらは広く一般に使われているウェブの構成要素である一方で、既存モデルに含まれていない要素である。
しかし、実際にこれらの要素は近年の成果として報告されている攻撃法
\textcolor{red}{
\cite{bcpattack}
}
にも利用されており、ウェブの安全性を評価する上で欠かすことのできない概念である。
本研究では、実装する時相論理を用いてこれらの要素の実装も行う。

本論文の構成は以下の通りである。
二章では、形式手法とウェブに関連する知識を述べる。
三章では、本研究の基礎となる既存のウェブセキュリティモデルとその問題点に述べる。
四章では、時相論理を包括したウェブセキュリティモデルを提案する。
五章では、現実に起こりうる事例を取り上げ検証結果を確認することで、提案モデルを評価する。

\end{document}
