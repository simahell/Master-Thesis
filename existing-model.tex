\documentclass[12pt,a4paper]{jbook}
\usepackage{mm-thesis}
\usepackage[dvipdfmx]{graphicx}
\usepackage{cite}
\usepackage{comment}
\usepackage{docmute}
\usepackage{color}
%\usepackage{amsmath}
%\usepackage{amsthm}
%\usepackage{amsfonts}

\begin{document}

\chapter{既存のウェブセキュリティモデル}
\section{セキュリティモデル}
\label{sec:SecurityModel}
セキュリティモデルは形式手法における入力にあたり、検査対象のシステムを命題論理を用いて表現したものである。
セキュリティモデルに記述する項目は主に以下の三つである。
\begin{itemize}
\item 対象のシステムの構造と動作
\item 脅威モデル(想定される攻撃者の能力)
\item 安全性要件(安全上満たしているべき条件)
\end{itemize}

\section{拡張を想定した汎用モデル}

\section{Cookieを包括するモデル}

\section{既存モデルにおける問題点}

\end{document}
